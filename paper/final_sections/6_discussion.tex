\section{Discussion and Conclusion}
\noindent
\par In this analysis, we explore both behavior and brain data. 
Our logistic model for the behavior 
data revealed clear disctinction between subject's aversion to loss 
(see appendix A.1). For example, we can see how subject 1 and 5 have 
very different behavior related to gain and loss. 
With the BOLD image analysis, we went through the process of designing
our design matrix for our linear model.
We realozed the huge influence on smoothing the data in the performance of
model. The noise modeling including the PCA could have been improved by 
adding other regressors to the model. Our first principal component
has a very big amplitude, despite our normalization. We used the filtered
data maked available by the OpenfMRI project to compare the brain
activation voxesls accross subjects. Although we did not have time to
further investigate these available data, that would have been interesting
to compare the performance of our resulting model and the filtered data.
Our first very exciting but yet challenging objective was to link the behavior
data with the brain image data. 
A very exciting but yest challenging objective was to link the behavior
data with the brain image data. The idea was to predict the subject response
on the [0:1] or [1,2,3,4] scale based on the activity of the significant voxel.
We started the process with the analysis of both behavior and BOLD data separetely.
Knowing the location of the significant voxels, one could for instance, include the
blood pressure level or 'level of activation' of the isolated voxels to predict
the bet of the corresponding subject. 






