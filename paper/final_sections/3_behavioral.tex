\section{Behavioral Data Analysis}
\subsection{Introduction}
\noindent
First of all, we generated some summary statistics including correlation among variables, and tried both linear and logistic regression analysis for behavioral data. The scientific questions that we have are - if we can predict response time and response (to gamble or not to gamble) based on the gain and loss.

\subsection{Methods}
\noindent
We did some explanatory data analysis and regression analysis using behavior 
data. For explanatory data analysis, we generated some summary statistics, 
including correlation among variables and simple plots to better understand 
the behavior. And then we used regression analysis to mainly answer two 
scientific questions. The scientific questions that we have are:
\begin{itemize}
\item If gain/loss would be significant for individuals who choose to 
participate and how much time it would take for them to respond.
\item If gain/loss would be significant for whether individuals would like to 
participate in the gamble. 
\end {itemize}

\subsubsection {Linear Regression}
I will change this.
\begin{enumerate}
\item  Response Time $\sim$ gain + loss
\item  Response Time $\sim$ diff(gain-loss)
\item  Response Time $\sim$ ratio(gain/loss)
\end {enumerate}

\subsubsection {Logistic Regression}
\noindent
To answer the second question, we fit logistic regression between Accept/Reject
Gamble and gain/loss. According to our analysis, the decision to whether take 
the gamble of most of subjects, in general, is more affected by loss amount 
rather than by gain amount.  For example, below is the analysis on the subject 
3. The regression line shows that it well follows the border between the two 
decisions: 1 (gamble) and 0 (not gamble). Right side of the line illustrates 
the decision to not gamble and it takes up more area relative to the opposite 
decision.


\subsection {Results}
I will update this





